\documentclass{article}
\usepackage{graphicx}
\usepackage{float}
\usepackage{titlesec}
\usepackage{datetime}
\usepackage{geometry}
\usepackage{minted}
\usepackage{xcolor}
\usepackage{listings}
\usepackage{caption}
\usepackage[document]{ragged2e}
\geometry{
 a4paper,
 left=25mm,
 top=25mm,
 }
\newdateformat{daymonthyear}{\THEDAY .\THEMONTH .\THEYEAR}
\title{
  \centering
  \includegraphics[width=\textwidth]{images/logo_PWr_kolor_poziom.png}\\
  \fontsize{28pt}{30pt}\selectfont Sprawozdanie 2\\
  \fontsize{14pt}{30pt}\selectfont Ćwiczenie 4.Oświetlenie scen}
\author{Krzysztof Zalewa}
\date{\daymonthyear\today}
\renewcommand*\contentsname{Spis treści}
\renewcommand{\listingscaption}{Fragment kodu}
\begin{document}
  \maketitle
  \pagebreak
  \tableofcontents
  
  \section{Wstęp teoretyczny}
  Lorem  ipsum  dolor  sit  amet,  consectetuer  adipiscing  
  elit.   Etiam  lobortisfacilisis sem.  Nullam nec mi et 
  neque pharetra sollicitudin.  Praesent imperdietmi nec ante. 
  Donec ullamcorper, felis non sodales...
  \subsection{Temat1}
  \subsubsection{Temat 1.1}

  \section{Zadanie laboratoryjne}
  \subsection{Treść zadania}
  \subsection{Opis działania programu}
  Zgodnie z treścią zadania program rysuje 4 obiekty. Domyślnie 
  jajko i czajnik rysowane są w kolorze czarnym. Jednakże jest 
  możliwość zmiany koloru na losowy. Wyświetlone obiekty można 
  obracać za pomocą klawiatury (Przycisk musi być wciśnięty i 
  przytrzymany).\linebreak 
  \textbf{Kontrola obrotu:}\linebreak  
  \textbf{A D} -obrót po osi Y\linebreak 
  \textbf{W S} - obrót po osi X\linebreak 
  \textbf{Q E} - obrót po osi Z\linebreak 
  \textbf{ESC} - Powrót do menu (okno konsolowe)\linebreak 
  \textbf{Ruch myszy w osi X} - Obrót kamery w osi X\linebreak 
  \textbf{Ruch myszy w osi Y} - Obrót kamery w osi Y\linebreak 
  \textbf{Scroll up} - Przybiliżenie obiektu\linebreak 
  \textbf{Scroll down} - Oddalenie obiektu\linebreak 
  \subsection{Kod programu}
  \begin{frame}
    \scriptsize
    \inputminted[
        style={vs},
        breaklines,
        breakanywhere, 
        linenos, 
        tabsize=4 
    ]{c++}{Lab4.cpp}
    \vspace{1em}
    \captionsetup{justification=centering}
    \captionof{listing}{Fragment kodu z programu}
    \label{lst:code}
  \end{frame}
    
  \section{Wnioski}
  Lorem  ipsum  dolor  sit  amet,  consectetuer  adipiscing  
  elit.   Etiam  lobortisfacilisis sem.  Nullam nec mi et 
  neque pharetra sollicitudin.  Praesent imperdietmi nec ante. 
  Donec ullamcorper, felis non sodales...
  
  \section{Źródła}
  Lorem  ipsum  dolor  sit  amet,  consectetuer  adipiscing  
  elit.   Etiam  lobortisfacilisis sem.  Nullam nec mi et 
  neque pharetra sollicitudin.  Praesent imperdietmi nec ante. 
  Donec ullamcorper, felis non sodales...
\end{document}