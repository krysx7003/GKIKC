\documentclass{article}
\usepackage{graphicx}
\usepackage{float}
\usepackage{titlesec}
\usepackage{datetime}
\usepackage{geometry}
\usepackage{placeins}
\usepackage{minted}
\usepackage{xcolor}
\usepackage{listings}
\usepackage{caption}
\usepackage[document]{ragged2e}
\usepackage[hidelinks]{hyperref}
\usepackage{enumitem}
\geometry{
 a4paper,
 left=25mm,
 top=25mm,
 }
\captionsetup{hypcap=false} 
\newdateformat{daymonthyear}{\THEDAY .\THEMONTH .\THEYEAR}
\title{
  \centering
  \includegraphics[width=\textwidth]{images/logo_PWr_kolor_poziom.png}\\
  \fontsize{28pt}{30pt}\selectfont Sprawozdanie 7\\
  \fontsize{14pt}{30pt}\selectfont Ćwiczenie 7.WebGL}
\author{Krzysztof Zalewa}
\date{\daymonthyear\today}
\renewcommand*\contentsname{Spis treści}
\renewcommand{\figurename}{Rysunek}
\renewcommand{\listingscaption}{Fragment kodu}
\begin{document}
    \maketitle
    \pagebreak
    \tableofcontents
    \FloatBarrier
    \section{Wstęp teoretyczny}
        \subsection{}
    \section{Zadanie laboratoryjne}
        \raggedright
        \subsection{Treść zadania}
            W ramach zadania laboratoryjnego należało napisać program kożystający z WebGL.
            Program ten miał wyświetlić dwa obiekty sześcia z brakującą ścianką i czworościan.
            Powinna być możliwość obrotu obiektów, wyboru który obiekt obracamy oraz zwiększenie
            lub zmniejszenie prędkości obrotu.
        \subsection{Opis działania programu}
            Program rysuje obiekty zgodnie z treścią zadania. W prawym górnym rogu znajdują się 
            przyciski które pozwalają na kontrolę obracania. Poniżej przycisków znajduje się suwak
            który pozwala na przyspieszenie lub zwolnienie animacji.
        \subsection{Kod programu}
            \begin{frame}
                \scriptsize
                \inputminted[
                    style={vs},
                    breaklines,
                    breakanywhere, 
                    linenos, 
                    tabsize=4 
                ]{html}{indexTex.html}
                \vspace{1em}
                \captionof{listing}{Fragment kodu z programu}
                \label{lst:html}
            \end{frame}
            \begin{frame}
                \scriptsize
                \inputminted[
                    style={vs},
                    breaklines,
                    breakanywhere, 
                    linenos, 
                    tabsize=4 
                ]{js}{./js/Lab7.js}
                \vspace{1em}
                \captionof{listing}{Fragment kodu z programu}
                \label{lst:main}
            \end{frame}
            \begin{frame}
                \scriptsize
                \inputminted[
                    style={vs},
                    breaklines,
                    breakanywhere, 
                    linenos, 
                    tabsize=4 
                ]{js}{./js/init-buffers.js}
                \vspace{1em}
                \captionof{listing}{Fragment kodu z programu}
                \label{lst:init}
            \end{frame}
            \begin{frame}
                \scriptsize
                \inputminted[
                    style={vs},
                    breaklines,
                    breakanywhere, 
                    linenos, 
                    tabsize=4 
                ]{js}{./js/draw-scene.js}
                \vspace{1em}
                \captionof{listing}{Fragment kodu z programu}
                \label{lst:draw}
            \end{frame}
    \section{Źródła}
        \begin{enumerate}[label=\arabic*.]
            \item 
        \end{enumerate}
\end{document}